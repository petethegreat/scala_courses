\part{Program Design}
\chapter{more with for}
\section{queries with for}

For expresions in scala can be likened to queries in an RDBMS.

\begin{lstlisting}
for (b <-books, a<- b.authors if a.startsWith "Bird," ) yield b.title

for { 
	b1 <- books
	b2 <- books
	if b1.title < b2.title
	a1 <- b1.authors
	a2 <- b2.authors
	if a1 == a2
	} yield a1
\end{lstlisting}

If the author has written 3 books, they will be printed 3 times.

\section{translating for-expressions}
For expressions are pretty handy. For expressions can generally be translated into expressions based on flatmap, map, and filter. Conversely, all of these functions can be defined in terms of for - 
\begin{lstlisting}
def mapFun[T,U](xs: List[T], f: T => U) : List[U] = for (x <- xs) yield f(x)

def flatMapFun[T,U]( xs:List[T], f: T => List[U]): List[U] = for (x <- xs, y <- f(x)) yield y

def filterFun[T](xs: List[T], f: T => Boolean): List[T] = for (x <-xs if f(x)) yield x

\end{lstlisting}

Scala translates for expressions into expressions based on map, flatmap and filter.

A really simple for-expression
\begin{lstlisting}
for (x <-e1) yield e2
\end{lstlisting}
can be translated to
\begin{lstlisting}
e1.map( x => e2)
\end{lstlisting}

Expressions of the form \lstinline|for (x <-e1 if f; s) yield e2|, where f is a filter and s is a (potentially empty) arbitrary sequence of generators and filters can be translated to 
\begin{lstlisting}
for (x<- e1.withFilter(x => f) ; s ) yield e2
\end{lstlisting}
withfilter is a lazy (i.e. smarter) implementation of filter. It does not create a new (intermediate) collection. The above expression is still contains a for expression, but we have removed one element (the if).

Cases containing more than one leading generator can be translated using flatmap
\begin{lstlisting}
for (x <-e1 ; y<-e2 ; s) yield e3
\end{lstlisting}
can be translated to 
\begin{lstlisting}
e1.flatMap(x => for (y <-e2 ; s) yield e3
\end{lstlisting}

In all of these cases, we are removing one element from the for expression. Thus an arbitrary expression can be reduced to a sequence of maps and flatmaps.

\begin{lstlisting}
for { 
	i <- 1 to N
	j <- 1 to i
	if isPrime(i + j)
} yield (i, j)
\end{lstlisting}

can be rewritten as
\begin{lstlisting}
//(1 until N).flatMap(i => for ( y <- 1 until i if isPrime(i,j)) yield (i,j) )

(1 until N).flatMap( i => 
	(1 until i).withFilter(j => isPrime(i + j)
	.map(j => (i,j)))
\end{lstlisting}

The for query above on books can be translated to \lstinline|books.flatmap(b => b.authors.withFilter(a => a.startswith("Bird").map(y => y.title)))|

Note that for expressions are not limited to lists//sequences/iterables. The translation only depends on the prescence of the methods map, flatmap, and withFilter. User defined types can be used in for expressions, provided these three methods are implemented.

For example, the collection books might instead be an interface to a database. Provided the methods are are implemented, for expressions can be used to query. the Scala database connection frameworks ScalaQuery and Slick make use of this.



\section{Monads}

monads must have an associated unit function, and have a flatmap method. I don't really get monads right now.

\section{Structural Induction}

Structural induction can be applied to trees.
To prove a property $P(t)$ for all trees of a certain type $t$
\begin{itemize}
	\item show that $P(l)$ holds for all leaves $l$ of a tree
	\item For each type of internal node t with subtrees $s_1, s_2 \ldots s_n$, show that
	$P(s_1)\wedge P(s_2)\wedge \ldots \wedge P(s_n)$ implies $P(t)$ 
\end{itemize}

if the property holds on all of the tree's subtrees, then it holds on the tree


consider the implementation of IntSets

\begin{lstlisting}
abstract class IntSet {
	def incl(x: Int) : IntSet
	def contains(x: Int): Boolean
}

object Empty extends IntSet {
	def contains(x: Int): Boolean = false
	def incl(x: Int): IntSet = NonEmpty(x, Empty, Empty)
}

case class NonEmpty(elem: Int, left: IntSet, right: IntSet) extends IntSet {
	def contains(x: Int): Boolean = 
		if (x < elem) left contains x
		else if (x > elem) right contains x
		else true
	def incl(x: Int): IntSet = 
		if  (x < elem) NonEmpty(elem, left incl x, right)
		else if (x > elem) NonEmpty(elem, left, right incl x)
		else this
}
\end{lstlisting}

how do we prove the correctness of this implementation holds? 
consider the following three laws (for integers x,y and Intset s):
\begin{itemize}
	\item \lstinline|Empty contains x = false|
	\item \lstinline| (s incl x) contains x = true|
	\item \lstinline|(s incl x) contains y = s contains y|
\end{itemize}

The first law is straightforward, and can be seen to be true (Empty.contains is false). 

For the second law (proposition), we can do strutural induction. Consider the base case when $s$ is an empty set. The we would like to show that \lstinline| (Empty incl x) contains x = true|. \lstinline|Empty incl x| evaluates to \lstinline| NonEmpty(x,...)|, and contains x will evaluate to true in this case.

For the base case when $s$ is NonEmpty, assume it takes the form NonEmpty(z,l,r), where l and r are subtrees. There are two cases to consider, z == x and z != x. If z == x, then NonEmpty(x,...) incl x will return this, which contains x, so we're good. if z != x, then NonEmpty(z) incl x will contain x, so we're good.
\section{streams}

Streams are like lists, but are evaluated lazily. The tail of the stream is not evaluated until it is needed. Streams are constructed from the object Stream.empty and the constructor stream.cons.

For lists, we might have something like
\begin{lstlisting}
def listrange(lo:Int, hi: Int): List[Int] = 
	if lo >= hi Nil
	else lo::listrange(lo+1,hi)
\end{lstlisting}
For streams, we would do
\begin{lstlisting}
def StreamRange(lo:Int, hi: Int): Stream[Int] = 
	if lo >= hi Stream.empty
	else Stream.cons(lo,StreamRange(lo+1,hi))
\end{lstlisting}

The standard shorthand for the cons operator, \lstinline|::|, will always produce a list. There is an equivalent for streams, the hash operator: \lstinline|#::|, which can be used in expressions and patterns. 

\section{State}
Up untill now we've been doing this purely functionally, as much as possible.
Some things will have a state. 

Everything witha mutable state will be constructed from variables. variables are declared with \lstinline|var| instead of \lstinline|val|. Variables can have their value changed later through assignment.

When assignment is possible, then determining whether or not things are equivalent becomes more difficult. Previously, if things evaluate to the same expression, then they are equal. 

\begin{lstlisting}
val x = E; val y = E;
val x = E; val y = x;
\end{lstlisting}
The two lines above produce the same result. In both cases, x and y evaluate to E
\begin{lstlisting}
val x = new BankAccount; val y = new BankAccount;
val x = new BankAccount; val y = x;
\end{lstlisting}
In this case, the two lines give different results. In the first, two new bankaccounts are created. In the second, y is copied from x.

How do we define "the same"? Operational equivalence - Execute the definitions of x and y, followed by an arbitrary set of operations involving x and y (S), observing all results. Then, execute the definitions followed by a different sequence of operations, S', in which every occurence of y in S has been replaced by x. If the results are different, then x and y are certainly different. Else, if every possible pair of sequences (S, S') are indistinguishable, then x and y are the same.

Assignment breaks the substitution model that we have been using up until now. In general, if we are not using purely functional code, then the substitution model will not hold.

\section{Loops}

Here's a possible definition of while that can be used to constrruct loops
\begin{lstlisting}

def WHILE(condition: => Boolean)(command :=> unit): unit = 
	if condition
	{ command
		WHILE(condition)(command)
		}
	else ()
\end{lstlisting}

 \section{event simulation}

 digital circuits - states are boolean.
 Will consider inverters (NOT), AND, and OR gates.

 Half adder - takes two inputs (A and B), and has two outputs (SUM and CARRY). CARRY is equal to A AND B, while SUM is A OR B AND NOT A AND B


\begin{figure}
\begin{circuitikz}
\draw 

(2,0) node[and port] (lowerand) {} 
(2,3) node[or port] (upperor) {}

(upperor.in 1) -- ++ (-2,0)
(lowerand.in 2) -- ++ (-2,0)

(upperor.out) ++( 2,0) node[and port, anchor=in 1](upperand){}
(lowerand.out) |- ++(1,1) node[not port](middlenot){}
(middlenot.out) |- (upperand.in 2)
(upperor.out) -- (upperand.in 1)

(upperor.in 1) -- ++ (-1.25,0) |- (lowerand.in 1)
(upperor.in 2) -- ++ (-0.75,0) |- (lowerand.in 2)
(lowerand.out) -- ++ (4,0)
;

\end{circuitikz}
\caption{half adder}
\end{figure}

two half adders can be combined (with an or gate) to form a full one-bit adder

\begin{figure}
\begin{circuitikz}
\draw 

(1,0) node[fourport, label=HA1] (ha1) {} 

(ha1.4) ++ (3,0) node[fourport,t=HA2, anchor=1] (ha2) {}
(ha2.2) ++ (1,0) node[or port,t=or, anchor=in 1] (or1) {}

(ha1.1) ++ (-1,0) node[ocirc](cin){Cin}
(ha1.4) ++ (-1,0) node[ocirc](a){A}
(ha2.4) ++ (-4,0) node[ocirc](b){B}
(ha2.3) ++ (3,0) node[ocirc](sum){Sum}
(or1.out) ++ (1,0) node[ocirc](carry){Cout}


(ha1.3) -- (ha2.1)
(ha2.2) -- (or1.in 1)
(ha1.2) -| (or1.in 2)

(cin) -- (ha1.1)
(a) -- (ha1.4)
(b) -- (ha2.4)
(ha2.3) -- (sum)
(or1.out) -- (carry)

;

\end{circuitikz}
\caption{full 1-bit adder}
\end{figure}
