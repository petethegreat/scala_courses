\documentclass{article}
\usepackage{graphicx}
% \usepackage{dot2texi}
\makeatletter
\@ifundefined{verbatim@out}{\newwrite\verbatim@out}{}
\makeatother
% \usepackage{tikz}
\usepackage{hyperref}
% \usetikzlibrary{shapes,arrows}
% \usepackage[pdf]{graphviz}
%\usepackage{feynmp}
\usepackage{subfigure}
\usepackage{dsfont}

\usepackage{listings}
\usepackage{color}

\definecolor{dkgreen}{rgb}{0,0.6,0}
\definecolor{gray}{rgb}{0.5,0.5,0.5}
\definecolor{mauve}{rgb}{0.58,0,0.82}

\lstset{frame=tb,
  language=Scala,
  aboveskip=3mm,
  belowskip=3mm,
  showstringspaces=false,
  columns=flexible,
  basicstyle={\small\ttfamily},
  numbers=none,
  numberstyle=\tiny\color{gray},
  keywordstyle=\color{blue},
  commentstyle=\color{dkgreen},
  stringstyle=\color{mauve},
  breaklines=true,
  breakatwhitespace=true,
  tabsize=3
}

\graphicspath{{figs/}}
\title{Scala Notes}

\author{Peter Thompson}

\begin{document}
\section{functional paradigms}


mutation - change some attribute while maintaining identity. For example, could define a polynomial class, then set a certain coefficient to a particular value.

functional programming - avoid mutation - use immutable variables (values)
take something and change it, resulting in something else.
minimise side effects

functional programming
restricted definition - no mutable variables, assignments, or imperative conttrol structures
wider sense - a functional programming language allows the construction of elegant programs that focus on fucntions

functions are first class functions

\section{misc}

\begin{lstlisting}
// Hello.java
import javax.swing.JApplet;
import java.awt.Graphics;

public class Hello extends JApplet {
    public void paintComponent(Graphics g) {
        g.drawString("Hello, world!", 65, 95);
    }    
}
\end{lstlisting}


\end{document}
