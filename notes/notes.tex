\documentclass{article}
\usepackage{graphicx}
% \usepackage{dot2texi}
\makeatletter
\@ifundefined{verbatim@out}{\newwrite\verbatim@out}{}
\makeatother
% \usepackage{tikz}
\usepackage{hyperref}
% \usetikzlibrary{shapes,arrows}
% \usepackage[pdf]{graphviz}
%\usepackage{feynmp}
\usepackage{subfigure}
\usepackage{dsfont}

\usepackage{listings}
\usepackage{color}

\definecolor{dkgreen}{rgb}{0,0.6,0}
\definecolor{gray}{rgb}{0.5,0.5,0.5}
\definecolor{mauve}{rgb}{0.58,0,0.82}

\lstset{frame=tb,
  language=Scala,
  aboveskip=3mm,
  belowskip=3mm,
  showstringspaces=false,
  columns=flexible,
  basicstyle={\small\ttfamily},
  numbers=none,
  numberstyle=\tiny\color{gray},
  keywordstyle=\color{blue},
  commentstyle=\color{dkgreen},
  stringstyle=\color{mauve},
  breaklines=true,
  breakatwhitespace=true,
  tabsize=3
}

\graphicspath{{figs/}}
\title{Scala Notes}

\author{Peter Thompson}

\begin{document}
\section{Functional Paradigms}
\subsection{Functional Programming}
Mutation is where an attribute of a variable can change while the identity of the variable is maintained.  For example, could define a polynomial class, then set a certain coefficient to a particular value.

Functional programming is a programming strategy that avoids mutation/reliance on state information. Immutable values are used. These can be transformed, but the idea is to minimise side effects. We don't want to pass an argument to a function that will then modify that argument. Input goes in, return value comes out with inputs unchanged.

The restricted definition of a functional programming framework is one in which there are no mutable variables, assignments, or imperative conttrol structures
In a wider sense, functional programming can be carried out in any language that allows the construction of elegant programs that focus on functions

In scala, functions are first class objects. They can be treated and passed around just like any other variable (as in python)


\subsection{call-by-name, call-by-value}
Functions arguments in scala can be handled two ways. The argument can be called by name, or called by value.

\begin{lstlisting}
//call by value
def CBVfunc(a:Int):Int = {...}
\end{lstlisting}
Function CBVfunc takes an int, returns an Int. The integer argument a is evaluated when the function is called.


\begin{lstlisting}
//call by name
def CBNfunc(a: =>Int): Int = {...}
\end{lstlisting}
Function CBNfunc takes an integer argment (a), which is evaluated when it needs to be (or not at all!).

Not all functions terminate, infinte loops are a thing

Both call by value and call by name will reduce to the same outcome provided
\begin{itemize}
  \item the reduced expression consists of pure functions (no state information/side effects?)
  \item both evaluations terminate (no infinte loops)
\end{itemize}

if call by value terminates, then call by name will also
converse is not guarenteed: call by name termination does not imply call by value termination
in call by name, unused arguments are not evaluated
\begin{itemize}
  \item could have a function that takes two arguments. The second argument is not used (always)
  \item could pass a non-terminating input to the CBN function, which is not used. no big deal
  \item call by value will try to evaluate it and get stuck
\end{itemize}

Below is an example of a function that will terminate when arguments are called by name but not when called by value.
\begin{lstlisting}
// an infinite loop, this run indefinitely when evaluated
def loop = loop 

// call by value function, arguments are evaluated once when the function is called
def mooseCBV(a: Int, b: Int):Int = a

// this evaluates loop, starting the infinte loop...
mooseCBV(1,loop)

// same as above, but using call-by-name (=>)
def mooseCBN(a: =>Int, b: => Int) = a

// returns 1, the loop is not evaluated
mooseCBN(1,loop)
\end{lstlisting}

Scala uses call by value by default, unless the function arguments are defined with \verb|=>|


\subsection{Conditionals}
Boolean operations don't always need to evaluate the right hand operand (short circuit evaluation)

\begin{verbatim}
true && e -> e
false && e -> false
true || e -> true
false || e -> e
\end{verbatim}

Things can be defined by name or by value.
so \lstinline|def x = loop| is a function, it is not evaluated untill it needs to be (call by name). 

\lstinline|val x = loop| evaluates to loop (call by value) immediately. this will kill your scala session/repl.

\subsection{Recursion}
Recursive functions must always have their return type explicitly defined (to make the compiler's life easier).

\subsubsection{Tail Recursion}
If a function calls itself as its last action, the stack frame can be reused
Essentially it acts the same as a loop

If a functions last action is to call a function, (maybe the same, maybe different function), then the stack frame can be used - this is a tail-call (tail recursion is recursive tail-calling).

Tail recursive factorial example(works)

\begin{lstlisting}
def factorial(N:Int) = {
    @scala.annotation.tailrec
    def currentProd(n:Int, prod:Int) :Int = {
        if (n==0) prod
        else currentProd(n-1,n*prod)
    }
    currentProd(N,1)
}
\end{lstlisting}

A lot of loops can be replaced by tail recursion. Usually this involves defining an inner function for the actual recursion, which accepts an accumulator argument in addition to other parameters. Recursive invocations of the inner function pass the current value of the accumulator, or return something when termination condition is met. For tail recursion (or recursion in general), it seems helpful to define the termination conditions at the very begining, then figure out the remaining logic.

\subsection{Blocks and Scope}
A block is defined by curly braces \lstinline|{}|

Definitiones inside a block are invisible outside the block. 
Definitions from outside the block are visible inside, provided they have not been shadowed.
A lot of object oriented functionality can be warngled from scopes and closures. Methods defined within a class constructor have acesss to the parameters of that constructor, even if those parameters are not assigned to fields of the class. 
For example:

\begin{lstlisting}
class NonEmpty(elem: Int, left: IntSet, right: IntSet) extends IntSet {
  def contains(x: Int): Boolean = if (x < elem) left.contains(x)
    else if (x > elem) right.contains(x)
    else true
    ...
\end{lstlisting}
the method \lstinline{contains} refers to \lstinline{elem}.

\subsection{higher order functions}
% TODO - clean below here
functions are first class values
they can be passed and returned
a function that does this is called a higher order function

this can be used to factor out common procedures. For example, 
\begin{lstlisting}
sumFunc(a:Int,b:Int,f: Int =>Int): Int = {
    if (a > b) 0 else f(a) + sumFunc(a+1,b,f)
}
\end{lstlisting}

defines a function sumFunc, that takes two integers and a function that takes an Int and returns an Int (\lstinline{Int => Int})
For example, we could sum all squares or cubes between 2 and 5 by calling
\begin{lstlisting}
sumFunc(2,5,square)
sumFunc(2.5.cube)
\end{lstlisting}
The notation \lstlinlne{A => B} is  a {\b function type}. it is a type that defines a mapping from type {\b A} to type {\b B} (by a function)

\subsection{anonymous functions}
strings exist as literals. We can just write "abc", and the compiler knows it to be a string. We don't need to do \lstinline{def str = "abc"; println(str)}
instead \lstinlineprintln{("abc")} works just fine
Same can be done with functions, we don't always need to define a function, we can define anonymous functions as needed. (same as lambda functions in python)

these are defined like this
\begin{lstlisting}
(x: Int) => x*x*x
\end{lstlisting}
the type of x can be omitted if it can be inferred.

anonymous functions are syntactic sugar:
\lstinline{(x:Int) = x*x} and \lstinline{def f(x:Int) = x*x; f} 
evaluate to the same.

tail recursive sum
\begin{lstlisting}
def sum(f: Int => Int, a:Int, b:Int) = {
    @scala.annotation.tailrec
    def doSum(total:Int, aval:Int):Int = {
        if (aval > b) total else doSum(total + f(aval),aval+1)
    }
    doSum(0,a)
}
\end{lstlisting}

\subsection{currying}
from docs -
\begin{verb} 
"Methods may define multiple parameter lists. When a method is called with a fewer number of parameter lists, then this will yield a function taking the missing parameter lists as its arguments."\end{verb}
a function that returns a function. Like a template in c++, for example

\begin{lstlisting}
def sum (f: Int => Int) :(Int, Int) => Int = {
    def sumF(a:Int,b:Int): Int = {
    if (a>b) 0 else f(a) + sumF(a+1,b)
    }
    sumF
}

\end{lstlisting}
sum now takes a function, and returns a function (return type is (Int,Int) => Int). the returned functino will take two Ints and return one. In this case, an when an initial function (f) is passed into sum, it will return a function that sums the initial function (f) within the supplied bounds.

so, we could do

def sumCubes = sum((x:Int) => x*x*x)
def sumSquares = sum((x:Int) => x*x)

and then call
sumSquares(2,3) // 13
sumCubes(4,7) // 748

Alternatively, we could invoke the returned function directly

val moose = sum((x:Int)=>x*x) (2,3) //13

Again, there is some syntactic sugar for currying

def sum(f:Int => Int) (a:Int,b:Int): Int = {
    if (a>b) 0 else f(a) + sum(f)(a+1,b)
}
is equivalent to the definitino of sum above, but without the definition of the inner function. 

multiple parameter lists
def f(args1)...(argsn) = E
where n > 1 is equivlaent to
def f(args1)...(argsn-1) = {def g(argsn) = E ; g}

or
def f(args1)...(argsn-1) = (argsn => E)

Carrying this through gives

def f = (args1 => (args2 => ...(argsn => E)))

named after Haskell Brooks Curry (same guy Haskell language is named after )



function that calculates product of values of function for points on an interval
write factorial in terms of product

function that generalises sum and product

def product(f: Int=>Int) (a:Int,b:Int) :Int = {
if (a>b) 1 else product(f)(a+1,b)*f(a)  
}

def factorial(n:int) = product((x:Int) =>x)(1,n)


def CumulativeFunctionOperation(operation: (Int,Int) => Int, initVal:Int)(f:Int=>Int)(a:Int,b:Int):Int = {
    if (a>b) initVal else operation(f(a),CumulativeFunctionOperation(operation,initVal)(f)(a+1,b))
}

def sum2:(Int=>Int)=>(Int,Int)=>Int = CumulativeFunctionOperation((x:Int,y:Int)=> x+y,0)

sum2(x=>x*x)(2,3) // 13

def prod2:(Int=>Int)=>(Int,Int)=>Int = CumulativeFunctionOperation((x:Int,y:Int)=> x*y,1)

prod2(x=>x*x)(2,3) // 36


CumulativeFunctionOperation is a form of map reduce
the operation is the reducer
the supplied funcion is the map
the bounds define the sequence that we are map/reducing

Mind Blown

These functions are in the worksheet "curying.sc" under misc_worksheets project

\section{misc}

% \begin{lstlisting}
% // Hello.java
% import javax.swing.JApplet;
% import java.awt.Graphics;

% public class Hello extends JApplet {
%     public void paintComponent(Graphics g) {
%         g.drawString("Hello, world!", 65, 95);
%     }    
% }
% \end{lstlisting}


\end{document}
